\documentclass[border = 10pt]{standalone}
\usepackage{tikz}
\usepackage[ngerman]{babel} %deutsche Trennung und neue Rechtschreibung
\usetikzlibrary{mindmap,trees}
\begin{document}
\pagestyle{empty}
\begin{tikzpicture}
\path[mindmap, concept color =  black, text =white]
node[concept] {VHDL}
[clockwise from = 0]
child[concept color=blue,sibling angle=60] {
node[concept] {Trivia}
[counterclockwise from = 0]
child[concept color=blue,sibling angle=60] {
node[concept] {Abk. für \underline{V}HSIC-HDL}
[clockwise from = 0]
child[concept color=blue,sibling angle=60] {
node[concept] {\underline{VHSIC} Very High Speed Integrated Circuits}
[clockwise from = 0]
}
child[concept color=blue,sibling angle=60] {
node[concept] {\underline{HDL}: Hardware Description Language}
[clockwise from = 0]
}
}
child[concept color=blue!90,sibling angle=60] {
node[concept] {Ziel der Entwicklung}
[clockwise from = 180]
child[concept color=blue!90,sibling angle=60] {
node[concept] {Unabhaengigkeit von verwendeter Technologie und Entwurfmethodik}
[clockwise from = 0]
}
child[concept color=blue!70, level distance = 17.36em,sibling angle=60] {
node[concept] {Beschreibung von Schaltungen auf versch. Abstraktionsebenen unter Verwendung versch. Sichtweisen}
[clockwise from = 0]
child[concept color=blue!70,sibling angle=60] {
node[concept] {Node Nr1}
[clockwise from = 0]
}
}
child[concept color=blue!90,sibling angle=60] {
node[concept] {Validierung mittels einzigen Simulator}
[clockwise from = 0]
}
}
}
;
\end{tikzpicture}\end{document};