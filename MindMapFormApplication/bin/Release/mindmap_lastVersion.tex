\documentclass[border = 10pt]{standalone}
\usepackage{tikz}
\usepackage[ngerman]{babel} %deutsche Trennung und neue Rechtschreibung
\usetikzlibrary{mindmap,trees}
\begin{document}
\pagestyle{empty}
\begin{tikzpicture}
\path[mindmap, concept color =black, text =white]
node[concept] {VHDL}
[clockwise from = 0]
child[concept color=red!70, level distance = 31.4em,sibling angle=68] {
node[concept] {Zeittafel}
[clockwise from = 0]
child[concept color=red!70,sibling angle=60] {
node[concept] {1983}
[clockwise from = 90]
child[concept color=red!70,sibling angle=90] {
node[concept] {VHDL 7.2}
[clockwise from = 0]
}
child[concept color=red!70,sibling angle=90] {
node[concept] {1987}
[clockwise from = 90]
child[concept color=red!70, level distance = 7em,sibling angle=90] {
node[concept] {Standardisierung von IEEE
als Standard 1076 (VHDL-87)}
[clockwise from = 90]
}
child[concept color=red!70,sibling angle=90] {
node[concept] {1993}
[clockwise from = 90]
child[concept color=red!70,sibling angle=90] {
node[concept] {Ueberarbeitung durch IEEE (VHDL-93)}
[clockwise from = 0]
}
child[concept color=red!70,sibling angle=90] {
node[concept] {2000}
[clockwise from = 90]
child[concept color=red!70,sibling angle=90] {
node[concept] {Weitere Ueberarbeitungen}
[clockwise from = 0]
}
child[concept color=red!70,sibling angle=90] {
node[concept] {2003}
[clockwise from = 90]
child[concept color=red!70,sibling angle=60] {
node[concept] {Erweiterung des Standards}
[counterclockwise from = 0]
child[concept color=blue,sibling angle=60] {
node[concept] {Modellierung rekonfigurierbarer Logik}
[clockwise from = 0]
}
child[concept color=blue, level distance = 7em,sibling angle=60] {
node[concept] {Integration von Aspekten der formalen Vertifikation}
[clockwise from = 0]
}
}
}
}
}
}
}
}
child[concept color=orange,sibling angle=68] {
node[concept] {Allgemeine Eigenschaften}
[clockwise from = 0]
child[concept color=orange,sibling angle=60] {
node[concept] {Jede gueltige VHDL}
[clockwise from = 0]
child[concept color=green,sibling angle=60] {
node[concept] {simulier-\\bar}
[clockwise from = 0]
}
child[concept color=yellow,sibling angle=60] {
node[concept] {nicht unbedingt synthetisierbar}
[clockwise from = 0]
}
}
child[concept color=red, level distance = 10.74em,sibling angle=60] {
node[concept] {Besonders bzgl. Zeitverhalten starke Abweichungen moeglich}
[clockwise from = 0]
}
}
child[concept color=blue,sibling angle=68] {
node[concept] {Trivia}
[counterclockwise from = 0]
child[concept color=blue,sibling angle=60] {
node[concept] {Abk. für \underline{V}HSIC-HDL}
[clockwise from = 0]
child[concept color=blue,sibling angle=60] {
node[concept] {\underline{VHSIC} Very High Speed Integrated Circuits}
[clockwise from = 0]
}
child[concept color=blue,sibling angle=60] {
node[concept] {\underline{HDL}: Hardware Description Language}
[clockwise from = 0]
}
}
child[concept color=blue!90,sibling angle=60] {
node[concept] {Ziel der Entwicklung}
[clockwise from = 180]
child[concept color=blue!90,sibling angle=60] {
node[concept] {Unabhaengigkeit von verwendeter Technologie und Entwurfmethodik}
[clockwise from = 0]
}
child[concept color=blue!70, level distance = 17.36em,sibling angle=60] {
node[concept] {Beschreibung von Schaltungen auf versch. Abstraktionsebenen unter Verwendung versch. Sichtweisen}
[clockwise from = 0]
}
child[concept color=blue!90,sibling angle=60] {
node[concept] {Validierung mittels einzigen Simulator}
[clockwise from = 0]
}
}
}
;
\end{tikzpicture}\end{document};